\pagestyle{fancy}
\fancyhf{}
\rhead{Introduction}
\lhead{Chapter 1}
\lfoot{\footnotesize{MES COE, B.E. COMPUTER YEAR 2021-22}}
\rfoot{\thepage}


\chapter{Introduction}


\section{Types}
Following are the different adaptive antenna algorithms \cite{ADR1}:
\begin{enumerate}
\item Least Mean Squares Algorithm. 
\item Sample Matrix Inversion Algorithm.
\item Recursive Least Square Algorithm.
\item Conjugate gradient method.
\item Constant Modulus Algorithm.
\end{enumerate}

\section{Background}

Adaptive Beamforming is a technique in which an array of antennas is exploited to achieve maximum reception in a specified direction by estimating the signal arrival from a desired direction (in the presence of noise) while signals of the same frequency from other directions are rejected.\\
The beamformer array output is given by, 
\begin{equation*}
y(t)= w^{H}\;x(t)
\end{equation*}
the optimum solution for the weight $w_{opt}$ is given by,\\
\begin{equation*}
w_{opt} = R^{-1}\;r
\end{equation*}
where,\; $r = E{[d^*(t)x(t)]}$ is the cross-correlation matrix between the desired signal
and the received signal,\; $R = E[x(t)x^{H}(t)]$ is the auto-correlation matrix of the received signal also known as the covariance matrix.


\section{Objectives}
\begin{itemize}
  \item Formulating the equation of the algorithm.
  \item Algorithm Analysis.
  \item Simulation using MATLAB.
\end{itemize}

